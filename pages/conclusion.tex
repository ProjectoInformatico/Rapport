\section{Conclusion}

        Le compilateur avec l'instruction manager se rapproche du fonctionnement d'un compilateur à la volée\footnote{\url{http://en.wikipedia.org/wiki/Just-in-time\_compilation}}. Un compilateur à la volée prend en paramètre un langage intermédiaire qui est transformé en bytecode puis peut être interprété directement. Les instructions émises dans l'instruction manager peuvent être assimilées à un langage intermédiaire. 
    
    On pourrait imaginer générer un langage intermédiaire compatible avec un compilateur JIT tel que Nanojit\footnote{\url{https://developer.mozilla.org/en-US/docs/Nanojit}}. On pourrait alors profiter de son optimisation de code intermédiaire.
        
