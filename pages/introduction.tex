
\section{Introduction}

    \subsection{Contexte et description}
    
    Le projet consiste à réaliser un compilateur de programme C-simplifié ainsi qu'un microprocesseur de type RISC avec pipe-line. Le compilateur est une application directe au cours d'Automates et Langages et le microprosseur au cours d'Architecture matérielle pour les systèmes informatiques.
    
    A l'issue de ce projet, nous sommes capable à partir d'un code source, de l'analyser, vérifier sa syntaxe afin de généré un byte-code pouvant être exécuté sur notre microprocesseur. 
    
    \subsection{Organisation}

    Le document est divisé en deux parties : le compilateur puis le microprocesseur. Chacune de ces parties décrit notre démarche, ce que nous avons réalisé ainsi que la manière dont nous l'avons fait et explique enfin les différentes difficultés rencontrées.

    Les codes sources cités dans le document sont disponibles en annexe.

    Les codes source du projet sont disponible en intégralité sur GitHub : 
    \begin{itemize}
    \item Compilateur : \url{https://github.com/ProjectoInformatico/Compilo}
    \item Interpréteur : \url{https://github.com/ProjectoInformatico/Interpreto}
    \item InstructionManager : \url{https://github.com/ProjectoInformatico/InstructionManager}
    \item Microprocesseur : \url{https://github.com/ProjectoInformatico/MicroProcesso}
    \item Présentation : \url{https://github.com/ProjectoInformatico/Slideshow}
    \item Rapport : \url{https://github.com/ProjectoInformatico/Rapport}
    \end{itemize}
